\documentclass[]{article}

%opening
\title{Statistics notes}
\author{José Guilherme Vanz}

\begin{document}

\maketitle

\begin{abstract}
My personal notes while I'm studying statistics
\end{abstract}

\section{Data Types}

\subsection{Categorical Data Type}
A group or a set of items
\paragraph{Hint} Think the values as labels for a group of items or individuals
\subsubsection{Ordinal (ordered)}
Values which are ranked are categorical ordinal data. 
\paragraph{Example} Classify how much do you like a dog. Boring, Cool, Amazing. 
\paragraph{Example} High, Med, Low
\paragraph{Example} Letter grade, Survey Rating. If you receive an A, this is higher than an A-. An A- is ranked higher than a B+, and so on...

\subsubsection{Nominal(no order)}
Categorical nominal data are values that do not have ranked order.
\paragraph{Example} Classify dog by the breed. Lab, pug, poodle
\paragraph{Example} Gender, Marital Status, Breakfast items. If you ate cereal, toast, eggs, or only coffee for breakfast; there is no rank ordering associated.

\subsection{Quantitative data type}
Numeric values that allow methematical operations.
\paragraph{Hint} If you can add,subtract, divide, multiple and get useful infomation. Probrably it's a quantitative data type
\subsubsection{Continous data}
Values that can be split into smaller values
\paragraph{Hint} Continuous data types are those that can take on decimal values
\paragraph{Example} Age of a dog. We can split the age by years, months, days, hours, minutes, seconds,...
%draw a picture of an examples
\paragraph{Example} 0.1, 0.2,1.5
\paragraph{Example} Height, Age, Income

\subsubsection{Discrete data}
Values that are countable
\paragraph{Example} 0,1,2,3,...
\paragraph{Example} Pages in a book, Trees in Yard, Dogs at a Coffee Shop


\subsubsection{Measures}
\paragraph{Aspects}
\begin{itemize}
\item center
\item spread
\item shape
\item outliers
\end{itemize}
\paragraph{Measure of spread}

Give an idea of how students differ
\paragraph{Measure of center (mean)}
Give an idea of the average student
\begin{itemize}
	\item[mean] sum of all values divided by the count of values $ \frac{sum}{count} $
	\item median
	\item mode
\end{itemize}


\section{Misc}

\paragraph{Percentille}

\section{Statistical association}

\subsection{Correlation}

Correlation is any statistical association.

\paragraph{Positive correlation}
A positive correlation is when two variables move to the same direction. For example, when a value from 
one variable increase the value of the other variable increase. In the same way, if a value of a variable
decrease, the value of the other variable decrease too.

For example, the length of an iron bar will increase as the temperature increases.

\paragraph{Negative correlation}
The negative correlation is when a variable value increase the other variable decrease and vice versa. 
For example, the volume of gas will decrease as the pressure increases

\paragraph{No correlation or zero correlation}
In this kind of correlation, when a variable value increase or decrease the other variable remains constant.
\end{document}
