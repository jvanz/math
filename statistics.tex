\documentclass[]{article}

%opening
\title{Statistics notes}
\author{José Guilherme Vanz}

\begin{document}

\maketitle

\begin{abstract}
My personal notes while I'm studying statistics
\end{abstract}

\part{Descriptive statistics}
\paragraph{Descriptive statistics} is about describing our collected data using the measures discussed in the following sections: measures of center, measures of spread, shape of the distrubution and outliers. Plot alse can be use to a better understanding of the data
\section{Data Types}

\section{Categorical Data Type}
A group or a set of items
\paragraph{Hint} Think the values as labels for a group of items or individuals
\subsection{Ordinal (ordered)}
Values which are ranked are categorical ordinal data. 
\paragraph{Example} Classify how much do you like a dog. Boring, Cool, Amazing. 
\paragraph{Example} High, Med, Low
\paragraph{Example} Letter grade, Survey Rating. If you receive an A, this is higher than an A-. An A- is ranked higher than a B+, and so on...

\subsection{Nominal(no order)}
Categorical nominal data are values that do not have ranked order.
\paragraph{Example} Classify dog by the breed. Lab, pug, poodle
\paragraph{Example} Gender, Marital Status, Breakfast items. If you ate cereal, toast, eggs, or only coffee for breakfast; there is no rank ordering associated.

\section{Quantitative data type}
Numeric values that allow methematical operations.
\paragraph{Hint} If you can add,subtract, divide, multiple and get useful infomation. Probrably it's a quantitative data type
\subsection{Continous data}
Values that can be split into smaller values
\paragraph{Hint} Continuous data types are those that can take on decimal values
\paragraph{Example} Age of a dog. We can split the age by years, months, days, hours, minutes, seconds,...
%draw a picture of an examples
\paragraph{Example} 0.1, 0.2,1.5
\paragraph{Example} Height, Age, Income

\subsection{Discrete data}
Values that are countable
\paragraph{Example} 0,1,2,3,...
\paragraph{Example} Pages in a book, Trees in Yard, Dogs at a Coffee Shop


\subsection{Measures}
There are four main aspects used to describe quantitative variables:
\begin{itemize}
\item Measures of \textbf{center}
\item Measures of \textbf{spread}
\item \textbf{Shape} of the distribution
\item \textbf{Outliers}
\end{itemize}


\subsubsection{Measure of center} Give an idea of the average student
\paragraph{mean} sum of all values divided by the count of values $ \frac{sum}{count} $
\paragraph{median} it's the middle values of a data set. This values can change depending if the number of values is odd or even. If it's odd, the median will be exact values in the center of the data set. For example, considering the following data set: 1, 2, 3, 3, 5, 8, 10. The mean is 3. 
However, if the number of elements in the data set is even, the median will be the \textbf{mean} of the values in the center. Considering the following data set: 1, 2, 3, 3, 5, 8, 10, 105. The median will be $ \frac{3+5}{2} = 4 $
%
\textbf{NOTE!} The previous examples in the \textbf{median} paragraphs consider an sorted data set.
\paragraph{mode} Most frequent value in a data set. For example: 1, 2, \underline{3}, \underline{3}, 5, 8, 10. The mode in this data set is 3
A data set can contains more than one mode. For example: 5, 8, \underline{15}, 7, \underline{10}, 22, 3, 1, \underline{15}, \underline{10}. If all values appear the same number of times, usually we say there is no mode.

\subsubsection{Measure of spread}Give an idea of how students differ
How far are points from one another

\paragraph{Range}
\paragraph{Interquartile range (IQR)}
\[ IQR = Q_{3} - Q_{1}\]
\paragraph{Variance} 
\[ \frac{1}{n}\sum_{i=1}^{n}(x_{i} - \bar{x})^2 \]
or considering the Bessel's correction
\[ \frac{1}{n-1}\sum_{i=1}^{n}(x_{i} - \bar{x})^2 \]

%TODO explain bessels correction

%https://stats.stackexchange.com/questions/3931/intuitive-explanation-for-dividing-by-n-1-when-calculating-standard-deviation

\paragraph{Standard deviation ($ \sigma $)}
The standard deviation is the square root of the variance. It is more used instenda of he variance because it shares the same units with the original data. While the variance has squared units. 
\[ \sigma = \sqrt{\frac{1}{n}\sum_{i=1}^{n}(x_{i} - \bar{x})^2} \]
or considering the Bessel's correction, also called the sample standard deviation ($ s $)
\[ s = \sqrt{\frac{1}{n-1}\sum_{i=1}^{n}(x_{i} - \bar{x})^2} \]

%TODO explain bessel's correction.
%https://classroom.udacity.com/nanodegrees/nd109/parts/a0060509-db24-41bf-9fb6-e5544550a8f3/modules/416cfbbe-053f-4cf2-bf12-4857a1db2476/lessons/5452179865/concepts/2308601260923
\subsubsection{Shape}

\paragraph{Histogram}
With histogram we can quicky identify the \textbf{shape} of our data. In general terms, the distribution of the data is associated with 3 shapes:

\begin{itemize}
	\item Symmetric(Normal) 
	Draw a image showing this sape
	\item Right-skewed 
	Draw a image showing this sape
	\item Left-swewed 
	Draw a image showing this sape
\end{itemize}


\begin{center}
	\begin{tabular}{| l | l | l |}
		\hline
		Shape & Mean vs. Median & Real world applications \\ \hline
		Symmetric(Normal) & Mean equals Median & Height, Weight, Errors, Precipitation \\ \hline
		Right-skewed & Mean greater Median & Amount of drugs remaining in a blood stream, time between phone calls at a call center, time intul light bulb dies \\ \hline
		Left-swewed & Mean less than Median & Grades as percentage in many universities, age of death, asset price changes \\ \hline
	\end{tabular}
\end{center}

Depending on the shape associated with our dataset, certain measures of center or spread may be better for summarizing our dataset.

When we have data that follows a normal distribution, we can completely understand our dataset using the mean and standard deviation.

However, if our dataset is skewed, the 5 number summary (and measures of center associated with it) might be better to summarize our dataset. 

% https://www.quora.com/What-are-some-real-world-examples-of-normally-distributed-quantities
% https://www.utdallas.edu/~scniu/OPRE-6301/documents/Important_Probability_Distributions.pdf
% https://stats.stackexchange.com/questions/89179/real-life-examples-of-distributions-with-negative-skewness
TODO: add a explanation about what is a bi-modal shape of a distribution

\subsubsection{Outliers}
\[ Outlier < Q_{1} - 1.5(IRQ) \]
\[ Outlier > Q_{3} + 1.5(IRQ) \]
 

\part{Inferential Statistics}
\paragraph{Inferential statistics} is about using our collected data to draw conclusions to a larger population

\begin{list}{}{}
	\item \textbf{population} - our entire group of interest
	\item \textbf{parameter} - numeric summary about a population
	\item \textbf{sample} - subset of the population
	\item \textbf{statistic} - numeric summary about a sample
\end{list}
\section{Misc}

\paragraph{Percentille}

\section{Statistical association}

\subsection{Correlation}

Correlation is any statistical association.

\paragraph{Positive correlation}
A positive correlation is when two variables move to the same direction. For example, when a value from 
one variable increase the value of the other variable increase. In the same way, if a value of a variable
decrease, the value of the other variable decrease too.

For example, the length of an iron bar will increase as the temperature increases.

\paragraph{Negative correlation}
The negative correlation is when a variable value increase the other variable decrease and vice versa. 
For example, the volume of gas will decrease as the pressure increases

\paragraph{No correlation or zero correlation}
In this kind of correlation, when a variable value increase or decrease the other variable remains constant.

\section{Notation}
\paragraph{Capital letter} signify random variables.
\paragraph{Random variable}
\paragraph{lowercase letter}When we look at individual instances of a particular random variable, we identify these as lowercase letters with subscripts attach themselves to each specific observation. 
\end{document}
