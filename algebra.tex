\documentclass[]{article}

%opening
\title{Algebra}
\author{José Guilherme Vanz}

\begin{document}

\maketitle

\begin{abstract}

\end{abstract}

\section{Natural numbers}

\textit{Natural numbers} start from 1 and goes forever always increasing by 1.

\[ 1,2,3,4,5... 1000, 1000000, ..., \infty\]

Negatives numbers, numbers with decimals, zero are \textbf{not} Natural numbers. For example:

\[ -1,000 \]
\[\frac{1}{3}\]
\[ 89.1 \]

\section{Integers}

All number from -$ \infty $ to $ \infty $. They can be written without any fractions or decimals

\[-4, -3, -2, -1, 0, 1, 2, 3, 4\]

\section{Set}

A \textbf{set} is a collection of numbers
A \textbf{subset} is a set of numbers that are all contained in another set

\section{Rational Numbers}

A rational number is a number that can be written as:

\[ \frac{a}{b} \]

where $a$ and $b$ are \textit{integers} and $b$ is different of zero $ b \neq 0$

% I'm not sure if this subsection should be here. But I'll keep it for now
\subsection{Ratio}
In Math a ratio between two numbers indicates how many times the first number contains the seconds. 

\end{document}
